% =========================================================================
% SciPost LaTeX template
% Version 1a (2016/06/14)
%
% Submissions to SciPost Journals should make use of this template.
%
% INSTRUCTIONS: simply look for the `TODO:' tokens and adapt your file.
%
% - please enable line numbers (package: lineno)
% - you should run LaTeX twice in order for the line numbers to appear
% =========================================================================


% TODO: uncommente ONE of the class declarations below
% If you are submitting a paper to SciPost Physics: uncomment next line
\documentclass{SciPost}
% If you are submitting a paper to SciPost Physics Lecture Notes: uncomment next line
%\documentclass[LectureNotes]{SciPost}
\usepackage{amsmath,amssymb,graphicx,bm,color,mathrsfs,verbatim,epstopdf,dcolumn,cancel}

%%%%%%%%%%%%%%%%%%%%%%%%%%%%%%%%%%%%%%%%%%%%%%%%%%%%%%%%%%%%%%%%

%hyperrefs
\usepackage{hyperref}
\hypersetup{ 
	colorlinks=true,
}


%define path for figs
\graphicspath{{../figs/}}
\usepackage{graphicx}
\usepackage{caption}
\usepackage{subcaption}

%%%%%%%%%%%%%%%%%%%%%%%%%%%%%%%%%%%%%%%%%%%%%%%%%%%%%%%%%%%%%%%%
%%%%%%%%%%%%%%%%%%%% Python code %%%%%%%%%%%%%%%%%%%%%%%%%%%%%%%
%%%%%%%%%%%%%%%%%%%%%%%%%%%%%%%%%%%%%%%%%%%%%%%%%%%%%%%%%%%%%%%%

\usepackage{listings}
% the following lines make sure the pdf code is copy-pastable
\usepackage{textcomp}
\usepackage[space=true]{accsupp}

\newcommand{\pdfactualhex}[3]{\newcommand{#1}{%
		\BeginAccSupp{method=hex,ActualText=#2}#3\EndAccSupp{}}}

\pdfactualhex{\pdfactualdspace}{2020}{\textperiodcentered\textperiodcentered}
\pdfactualhex{\pdfactualsquote}{27}{'}
\pdfactualhex{\pdfactualbtick}{60}{`}

% define colours 
\definecolor{deepblue}{rgb}{0,0,0.8}
\definecolor{deepred}{rgb}{1.0,0,0}
\definecolor{deepgreen}{rgb}{0,0.7,0}
\definecolor{blueviolet}{RGB}{138,43,226}
\definecolor{darkyellow}{RGB}{204,204,0}
\definecolor{codegray}{rgb}{0.6,0.6,0.6}
\definecolor{weborange}{RGB}{255,165,0}
\definecolor{gold}{RGB}{255,205,0}
\definecolor{codegreen}{rgb}{0,0.6,0}
\definecolor{codepurple}{rgb}{0.58,0,0.82}

%\definecolor{backcolour}{rgb}{0.0,0.0,0.0}
\definecolor{backcolour}{rgb}{0.95,0.95,0.92}

\lstdefinestyle{sublime}{
	backgroundcolor=\color{backcolour},   
	commentstyle=\color{deepgreen},
	keywordstyle=\color{deepred},
	numberstyle=\tiny,
	stringstyle=\color{weborange},
	basicstyle=\small\ttfamily, %\footnotesize,
	breakatwhitespace=false,         
	breaklines=true,                 
	captionpos=t,                    
	keepspaces=true,                 
	numbers=left,                    
	numbersep=5pt,                  
	showspaces=false,                
	showstringspaces=false,
	showtabs=false,                  
	tabsize=4,
	columns=flexible,
	emptylines=10000,
	literate={'}{\pdfactualsquote}1{`}{\pdfactualbtick}1{\ \ }{\pdfactualdspace}2
}
\lstset{style=sublime,language=Python,inputpath=../scripts/,keywords={lambda,xrange,abs,for,return},breaklines}

% change default listings caption title
\renewcommand{\lstlistingname}{\qspin\ \emph{Example Code}}% Listing -> q\spin\ Example Code


%%%%%% the following lines put the slashed zero in the code environtmnet listings

\usepackage{marvosym,etoolbox}
% this replaces 0 with \0 in lstings
\lstset{literate={0}{\0}1{0\ }{\0\ }2}

\renewcommand*\ttdefault{txtt}
\usepackage[T1]{fontenc}
\usepackage{graphicx}
% defines \0 as mirro of 0
\newcommand\0{\scalebox{-1}[1]{0}}
% fix for \texttt and \ttfamily
\let\svttfamily\ttfamily
\let\svtexttt\texttt
\catcode`0=\active
\def0{\0}
\renewcommand\ttfamily{\svttfamily\catcode`0=\active }
\renewcommand\texttt{\bgroup\ttfamily\texttthelp}
\def\texttthelp#1{#1\egroup}
\catcode`0=12 %

%%%%%%%


%%%%%%%%%%%%%%%%%%%%%%%%%%%%%%%%%%%%%%%%%%%%%%%%%%%%%%%%%%%%%%%%
%%%%%%%%%%%%%%%%%%%%% qspin logo %%%%%%%%%%%%%%%%%%%%%%%%%%%%%%%
%%%%%%%%%%%%%%%%%%%%%%%%%%%%%%%%%%%%%%%%%%%%%%%%%%%%%%%%%%%%%%%% 

\usepackage{upgreek}
\newcommand{\qspin}{$\mathcal{Q}^{\mathrm{u}}\!\mathcal{S}\uprho\mathrm{\text{\textexclamdown}}\mathcal{N}$}


%%%%%%%%%%%%%%%%%%%%%%%%%%%%%%%%%%%%%%%%%%%%%%%%%%%%%%%%%%%%%%%%
%%%%%%%%%%%%  vertical text on the right %%%%%%%%%%%%%%%%%%%%%%%
%%%%%%%%%%%%%%%%%%%%%%%%%%%%%%%%%%%%%%%%%%%%%%%%%%%%%%%%%%%%%%%%
\usepackage{background}
\usepackage{geometry}

\definecolor{textcolor}{HTML}{0A75A8}
\newcommand\Text{ \emph{to report a bug pls visit https://github.com/weinbe58/qspin/issues} }

\SetBgColor{textcolor}
\SetBgOpacity{0.5}
\SetBgAngle{-90}
\SetBgPosition{current page.center}
\SetBgVshift{0.35\textwidth}
\SetBgScale{1.8}
\SetBgContents{\sffamily\Text}
%%%%%%%%%%%%%%%%%%%%%%%%%%%%%%%%%%%%%%%%%%%%%%%%%%%%%%%%%%%%%%%% 

%\usepackage{ulem}

\newcommand*{\red}{\textcolor{red}}
\newcommand*{\blue}{\textcolor{blue}}
\newcommand*{\cyan}{\textcolor{cyan}}
\newcommand*{\green}{\textcolor{green}}



\begin{document}
% TODO: write your article's title here. 
% The article title is centered, Large boldface, and should fit in two lines
\begin{center}{\Large \textbf{
\qspin: a Python Package for Dynamics and Exact Diagonalisation of Quantum Many Body Systems\\
\large part I: spin chains
}}\end{center}

% TODO: write the author list here. Use initials + surname format.
% Separate subsequent authors by a comma, omit comma at the end of the list.
% Mark the corresponding author with a superscript *. 
\begin{center}
Phillip Weinberg\textsuperscript{*} and Marin Bukov
\end{center}

% TODO: write all affiliations here. 
% Format: institute, city, country
\begin{center}
Department of Physics, Boston University, \\
590 Commonwealth Ave., Boston, MA 02215, USA
\\
% TODO: provide email address of corresponding author
* weinbe58@bu.edu
\end{center}

\begin{center}
\today
\end{center}

% For convenience during refereeing: line numbers
%\linenumbers

\section*{Abstract}
{\bf 
We present a new open-source Python package for quantum dynamics of spin chains based on exact diagonalisation, called \qspin. The package is well-suited to study, among others, quantum quenches at finite and infinite times, the Eigenstate Thermalisation hypothesis, many-body localisation and other dynamical phase transitions, periodically-driven (Floquet) systems, adiabatic and counter-diabatic ramps, and spin-photon interactions. Moreover, \qspin's user-friendly interface can easily be used in combination with other Python packages which makes it amenable to a high-level customisation. We explain how to use \qspin\ using three detailed examples: (i) adiabatic ramping of parameters in the many-body localised XXZ model, (ii) heating in the periodically-driven transverse-field Ising model in a parallel field, and (iii) quantised light-atom interactions: recovering the periodically-driven atom in the semi-classical limit of a static Hamiltonian.
}


% TODO: include a table of contents (optional)
% Guideline: if your paper is longer that 6 pages, include a TOC
% To remove the TOC, simply cut the following block
\vspace{10pt}
\noindent\rule{\textwidth}{1pt}
\tableofcontents\thispagestyle{fancy}
\noindent\rule{\textwidth}{1pt}
\vspace{10pt}


\section{What Problems can I Solve with \qspin?}
\label{sec:intro}

%\cite{SciPy_package}

% TODO: 
% Provide your bibliography here. You have two options:
%\bibliographystyle{SciPost_bibstyle}
%\bibliographystyle{abbrv}
% FIRST OPTION - write your entries here directly, following the example below, including Author(s), Title, Journal Ref. with year in parentheses at the end, followed by the DOI number.
%\begin{thebibliography}{99}
%\bibitem{1931_Bethe_ZP_71} H. A. Bethe, {\it Zur Theorie der Metalle. i. Eigenwerte und Eigenfunktionen der linearen Atomkette}, Zeit. f{\"u}r Phys. {\bf 71}, 205 (1931), \doi{10.1007\%2FBF01341708}.
%\bibitem{arXiv:1108.2700} P. Ginsparg, {\it It was twenty years ago today... }, \url{http://arxiv.org/abs/1108.2700}.
%\end{thebibliography}

% SECOND OPTION:
% Use your bibtex library
% \bibliographystyle{SciPost_bibstyle} % Include this style file here only if you are not using our template
%\bibliography{qspin_test}

\nolinenumbers

\end{document}
